% Use the command pdflatex hello.tex on cmd to create the pdf file

\documentclass[12pt,a4paper]{article}
\usepackage[utf8]{inputenc} % Codificação de caracteres
\usepackage[brazil]{babel} % Suporte para Português Brasil
\usepackage{amsmath}        % Para fórmulas matemáticas
\usepackage{geometry}       % Para definir as margens
\usepackage{times}          % Usa a fonte Times, geralmente aceita pelas normas ABNT

% Definição das margens de acordo com a ABNT
\geometry{
a4paper,
left=3cm,
right=2cm,
top=3cm,
bottom=2cm
}

\begin{document}
\begin{titlepage} % Cria uma página de título
\begin{center}
\vspace*{\fill} % Centraliza verticalmente o título
\Huge Introdução à Matemática % \Huge é o maior tamanho de fonte disponível
\vspace*{\fill}
\end{center}

\begin{flushright}
% Adiciona o nome no canto inferior direito
\normalsize André Luiz Roberti Cardoso % \normalsize para voltar ao tamanho normal de fonte
\end{flushright}

\thispagestyle{empty} % Remove a numeração da página
\end{titlepage}

% Insere o sumário
\tableofcontents
\newpage % Começa o conteúdo em uma nova página após o sumário

\section{Introdução} % Seção principal

\subsection{Frações e Equações} % Subseção dentro de Introdução

\subsubsection{Axiomas} % Subsubseção dentro de Frações e Equações

Toda a matemática se baseia em Axiomas. \\
Axiomas são verdades que não precisam ser provadas. A partir deles, podemos deduzir outras verdades.\\
Cada área da matemática deve respeitar seus próprios axiomas.

\begin{enumerate}
    \item Coisas iguais às mesmas coisas, são iguais entre si.
    \[
    a = b \text{ e } b = c \text{, então } a = c.
    \]
    \item Se somarmos ou subtrairmos a mesma quantidade de ambos os lados de uma equação, o resultado será uma equação verdadeira.
    \[
    a = b \text{, então } a + c = b + c.
    \]
\end{enumerate}

\subsubsection{Frações}
Todo número escrito como $\frac{a}{b}$ é uma fração.

\begin{enumerate}
    \item Adição e Subtração de Frações
    \[
    \frac{a}{b} \pm \frac{c}{d} = \frac{ad \pm bc}{bd}.
    \]
    \item Multiplicação de Frações
    \[
    \frac{a}{b} \times \frac{c}{d} = \frac{ac}{bd}.
    \]
    \item Divisão de Frações
    \[
    \frac{a}{b} \div \frac{c}{d} = \frac{a}{b} \times \frac{d}{c}.
    \]
    \[
    Prova:
        \frac{a}{b} \div \frac{c}{d}
        = \frac{a}{b} \times d \times \frac{1}{c} \div \frac{c}{d} \times d \times \frac{1}{c}
        = \frac{ad}{bc}
    \]
\end{enumerate}

\clearpage

\subsection{Fatoração e Manipulações Algébricas}
\subsubsection{Axiomas Adição e Subtração}
\begin{enumerate}
    \item Comutatividade
    \[
    a + b = b + a.
    \]
    \item Associatividade
    \[
    a + (b + c) = (a + b) + c.
    \]
    \item Elemento Neutro
    \[
    a + 0 = a.
    \]
    \item Elemento Oposto
    \[
    a + (-a) = 0.
    \]
\end{enumerate}

\subsubsection{Axiomas Multiplicação e Divisão}
\begin{enumerate}
    \item Comutatividade
    \[
    a \times b = b \times a.
    \]
    \item Associatividade
    \[
    a \times (b \times c) = (a \times b) \times c.
    \]
    \item Elemento Neutro
    \[
    a \times 1 = a.
    \]
    \item Elemento Oposto
    \[
    a \times \frac{1}{a} = 1.
    \]
\end{enumerate}

\subsubsection{Fatoração}
\begin{enumerate}
    \item
    \[
        a(b+c) = ab + ac.
    \]
    \item
    \[
    (a+b)^2 = a^2 + 2ab + b^2.
    \]
    \item
    \[
        (a-b)^2 = a^2 - 2ab + b^2.
    \]    
    \item
    \[
    a^2 - b^2 = (a + b)(a - b).
    \]
    \item
    \[
    a^3 + b^3 = (a + b)(a^2 - ab + b^2).
    \]
    \item
    \[
    a^3 - b^3 = (a - b)(a^2 + ab + b^2).
    \]
\end{enumerate}

\subsubsection{Manipulações de Equações}
\begin{enumerate}
    \item Multiplicar por um fator
    \begin{enumerate}
        \item 
        \[ 
        5x + y = 3 \quad \times 10
        \]
        \item 
        \[
        50x + 10y = 30
        \]
        \item 
        \[
        10x + 2y = 6 \quad \div 5
        \]
    \end{enumerate}
\end{enumerate} % Finaliza a enumeração para "Multiplicar por um fator"

% Começa um novo item para "Somar/Subtrair Equações"
\item Somar/Subtrair Equações
\begin{align*}
    \left\{
    \begin{array}{rcl}
        5x + 2y &=& 3 \\
        3x - 4y &=& 5
    \end{array}
    \right.
\end{align*}

\end{document}

